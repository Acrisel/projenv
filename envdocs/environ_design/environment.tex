\documentclass[accord_design.tex]{subfiles}
\begin{document}
\chapter{Environment}\index{Environment}
Environment is a special type of dictionary holding parameters used by parts. As figure Figure~\ref{fig:environment_tree} (page~\pageref{fig:environment_tree}) shows, there are multiple levels and types of environments playing part in program run.

Environment evaluation process walks the project tree from top to program location.  In each node (package branch), there are three files defining nodes environment:

\section{Definitions}
\setcounter{tablerownumbers}{0}
\begin{longtable}{|r|l|p{7cm}|}
	\hline
	\emph{No.} & \emph{Name} & \emph{Description} \\ \hline
	\rownumber & .projectenv\index{environ!.projectenv} & read-only project level environment variables; also identifies project root location. \\ \hline
	\rownumber & projectenv\index{environ!projectenv} & addon environment for given read-only environment. \\ \hline
	\rownumber & personalenv\index{environ!personalenv} & personal overrides for project or package environment.  It may include only overrides. \\ \hline
	\rownumber & .packageenv \index{environ!.packageenv}& read-only package level environment variables. \\ \hline
	\rownumber & packageenv \index{environ!packageenv}& addon environment for given read-only environment in local or upper environments. \\ \hline
\caption{Environment Definitions}
\label{table:environ-defs}
\end{longtable}


\section{Program Interface}
Within programs there are three types of access points to environment variables.
To get environ dictionary, program can perform the following command:
\begin{enumerate}
	\item Loading environment variables from project structure
	\item Updating environment variable in program
	\item Accessing environment variables
\end{enumerate}

\subsection{Loading environmentvariables}\index{Environment!Loading}\index{Environment!Environ}

	\begin{lstlisting}
	import environ 
	env=environ.Environ()
	\end{lstlisting}

When Program evaluates environment, it starts with root location going down the tree up to an including package environment of Program location. 

Environ \_\_init\_\_ has the following signature:
\textbox{Environ(env=None, path=None, envtree=True, osenv=True, syntax='XML', config=None\emph{dict})}

\begin{table}[!htb]
\setcounter{tablerownumbers}{0}
\begin{tabularx}{\linewidth}{|r|X|X|l|}
	\hline
	\emph{No.} & \emph{Name} & \emph{Description} & \emph{Default Values} \\ \hline
	\rownumber & env & Direct environment list to override environment tree. & None \\ \hline
	\rownumber & path & Path to project location or to package within. & Current working directory \\ \hline
	\rownumber & envtree & If set, load enviroenment files from directory tree. & True \\ \hline
	\rownumber & osenv & If set, load os environ. & True \\ \hline
	\rownumber & syntax & Expected structure of environment files. & XML \\ \hline
	\rownumber & config\index{environ!config} & dictionary overriding default names for environment files. & '.projectenv':'.projectenv' \\ \cline{4-4} 
	& & & 'packageenv':'packageenv' \\ \cline{4-4} 
	& & & 'personalenv':'personalenv' \\ \cline{4-4}
	& & & 'envtag':'environ' \\ \hline
\end{tabularx}
\caption{Environ signature arguments}
\label{table:environ-args}
\end{table}

In addition to file name overrides, configuration dictionary can provide syntax used in these files and the environment root tag under which environment variables are defined. 

Within derivative articles environment can be updated with environment variable as follows:

\subsection{Updating environment variables}\index{Environment!Update}\index{Environment!EnvVar}
\begin{lstlisting}
env.update([
  EnvVar(name='REJ_ALLOWED',cast='integer',value=0,input=True),
  EnvVar(name='OUT_FILE',value='${VAR_LOC}/summary.csv',cast='path', input=True),
  EnvVar(name='RATE',override='True',cast='integer',value=5,input=True)])
\end{lstlisting}

\paragraph{input} flag will set the variable with the value of parent environment, if defined.
\paragraph{override} flags environment variable as changeable by derivative program articles.

\subsection{Accessing environment variables}
\begin{lstlisting}
ofile=env.environ[]'OUT_FILE']
rate=env.get('RATE')
\end{lstlisting}

In the first case, direct access, KeyError exception may be sent if variable name does not exist.

In the second example, None value will be returned if not found. 


\section{Single Project Environment Tree}
At each node, environment is evaluated in the following sequence:
\begin{enumerate}
	\item First .projectenv, if available, is read and set.
	\item Next packageenv, if available, is read and set.
	\item Finally, personalenv overrides, if available, is read and set.
\end{enumerate} 

\begin{figure}[h]
	\centering
	\framebox[\textwidth]{
		\begin{minipage}{0.9\textwidth}
			\dirtree{%
				.1 Project.
				.2 .projectenv.
				.2 packageenv.
				.2 personalenv.
				.2 Program A.
				.2 Package A.
				.3 packageenv.
				.3 personalenv.
				.4 Package AB.
				.5 packageenv.
				.5 personalenv.
				.5 Program AB.
			}
		\end{minipage}
	}
	\caption{Environment tree example}
	\label{fig:environment_tree}
\end{figure}

Figure~\ref{fig:environment_tree} (page~\pageref{fig:environment_tree}) shows example environment tree in a project.

When the above command is engaged in Program A, it would include environment setting of Project and Package A locations.  Program AB will include Program A, Package A and Package AB accordingly.

\lstset{
	basicstyle=\ttfamily\footnotesize,
	columns=fullflexible,
	showstringspaces=false,
	numbers=left,                   % where to put the line-numbers
	numberstyle=\tiny\color{gray},  % the style that is used for the line-numbers
	stepnumber=1,
	numbersep=5pt,                  % how far the line-numbers are from the code
	backgroundcolor=\color{white},      % choose the background color. You must add \usepackage{color}
	showspaces=false,               % show spaces adding particular underscores
	showstringspaces=false,         % underline spaces within strings
	showtabs=false,                 % show tabs within strings adding particular underscores
	frame=none,                   % adds a frame around the code
	rulecolor=\color{black},        % if not set, the frame-color may be changed on line-breaks within not-black text (e.g. commens (green here))
	tabsize=2,                      % sets default tabsize to 2 spaces
	captionpos=b,                   % sets the caption-position to bottom
	breaklines=true,                % sets automatic line breaking
	breakatwhitespace=false,        % sets if automatic breaks should only happen at whitespace
	title=\lstname,                   % show the filename of files included with \lstinputlisting;
	% also try caption instead of title  
	commentstyle=\color{gray}\upshape
}


\lstdefinelanguage{XML}
{
	morestring=[s][\color{mauve}]{"}{"},
	morestring=[s][\color{black}]{>}{<},
	morecomment=[s]{<?}{?>},
	morecomment=[s][\color{dkgreen}]{<!--}{-->},
	stringstyle=\color{black},
	identifierstyle=\color{DarkBlue},
	keywordstyle=\color{red},
	morekeywords={xmlns,xsi,noNamespaceSchemaLocation,type,id,x,y,source,target,version,tool,transRef,roleRef,objective,eventually,name,value,cast,export,input, environ}% list your attributes here
}
%\lstinputlisting[language=XML]{'sources/environ_example_1.xml'}

\begin{lstlisting}[language=XML, label=lst:project_env, caption=Example for project environment file]
<environment>
  <environ>
    <var name='AC_WS_LOC' value='${HOME}/sand/myproject' export='True'/>
    <var name='AC_ENV_NAME' value='test' export='True'/>
    <var name='AC_VAR_BASE' value='${HOME}/var/data/' export='True'/>
    <var name='AC_LOG_LEVEL' value='DEBUG' export='True'/> 
    <var name='AC_LOG_STDOUT' value='True' override='True' export='True' cast='boolean'/>
    <var name='AC_LOG_STDOUT_LEVEL' value='INFO' override='True' export='True'/>
    <var name='AC_LOG_STDERR' value='True' override='True' export='True' cast='boolean'/>
    <var name='AC_LOG_STDERR_LEVEL' value='CRITICAL' override='True' export='True'/>
  </environ>
</environment>
\end{lstlisting}

Listing \ref{lst:project_env} shows example of an environment file.  Core environment is tagged under \emph{\textless~environ\textgreater}.  Environ mechanism would look for this tag.  Once found, it would evaluate its content as environment directive.

\paragraph{Note:} \emph{\textless~environment\textgreater} tag is to provide enclosure to environ.  Environ mechanism is not depending on its existent per se.  However, some kind on enclosure is required;  \emph{\textless~environ\textgreater} can not be in top level of the XML.

\section{Multiple Project Environment Tree}
At each import, environment is evaluated in the following sequence:
\begin{enumerate}
	\item First get the node representation of imported path.
	\item Evaluate it recursively (loading imports).
	\item Finally, insert the resulted imported map instead of the import directive (flat).
\end{enumerate} 

%\lstinputlisting[language=XML]{'sources/environ_example_1.xml'}

\begin{lstlisting}[language=XML, label=lst:env_proja, caption='Project A: /Users/me/projs/proja/.projectenv.xml]
<environment>
  <environ>
    <var name='FILE_LOC' value='/Users/me/tmp/' export='True'/>
    <var name='FILE_NAME' value='aname' export='True'/>
    <var name='FILE_PATH' value='${FILE_LOC}${FILE_NAME}' export='True'/>
  </environ>
</environment>
\end{lstlisting}

\begin{lstlisting}[language=XML, label=lst:env_projb, caption='Project B: /Users/me/projs/projb/.projectenv.xml']
<environment>
  <environ>
    <import name='proja' path='/Users/me/projs/proja/.projectenv.xml'/>
    <var name='FILE_NAME' value='bname' export='True'/>
  </environ>
</environment>
\end{lstlisting}

Listings \ref{lst:env_projb} shows import project directive within project B's environment.  In project B's context, FILE\_PATH will result with the value \emph{/Users/me/tmp/bname}.

\paragraph{Recursive} inclusion of environments (recursive import statement) would cause evaluation of environment variables to be loaded recursively.  Consideration is given to overrides in post import environments.

\paragraph{Note:}import path can only include environment variables that are in the OS level pre-eveluation.


\end{document}