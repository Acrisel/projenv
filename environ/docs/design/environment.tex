\documentclass[design.tex]{subfiles}
\begin{document}
\chapter{Environment}\index{Environment}
Environment is a special type of dictionary holding parameters used by parts. As figure Figure~\ref{fig:environment_tree} (page~\pageref{fig:environment_tree}) shows, there are multiple levels and types of environments playing part in program run.

Environment evaluation process walks the project tree from top to program location.  In each non-root node (package branch), there could be one file defining nodes of environment. In root note there could be two types. In addition ability overwrite variables in personal setting is supported using personalenv file.

\section{Definitions}
\setcounter{tablerownumbers}{0}
\begin{longtable}{|r|l|p{7cm}|}
	\hline
	\emph{No.} & \emph{Name} & \emph{Description} \\ \hline
	\rownumber & .projectenv\index{environ!.projectenv} & read-only project level environment variables; also identifies project root location. \\ \hline
	\rownumber & packageenv \index{environ!packageenv}& addon environment for given read-only environment in local or upper environments. \\ \hline
	\rownumber & personalenv \index{environ!personalenv}& personal overrides for project or package environment. It may include only overrides. \\ \hline
\caption{Environment Definitions}
\label{table:environ-defs}
\end{longtable}


\section{Program Interface}
Within programs there are three types of access points to the environment variables.
To get projenv dictionary, program can perform the following command:
\begin{enumerate}
	\item Loading environment variables from project structure
	\item Updating environment variable in program
	\item Accessing environment variables
\end{enumerate}

\subsection{Loading environment variables}\index{Load}\index{Environ}

	\begin{lstlisting}
	import projenv 
	env=projenv.Environ()
	\end{lstlisting}

When Program evaluates environment, it starts with root location going down the tree up to and including package environment of Program location. 

Environ class \_\_init\_\_ has the following signature:
\textbox{Environ(self, osenv=True, configure=None, trace\_env=None, logclass=None, logger=None)}

\begin{table}[!htb]
\setcounter{tablerownumbers}{0}
\begin{tabularx}{\linewidth}{|r|X|X|l|}
	\hline
	\emph{No.} & \emph{Name} & \emph{Description} & \emph{Default Values} \\ \hline
	\rownumber & osenv & If set, load os environ. & True \\ \hline
	\rownumber & configure\index{environ!configure} & dictionary overriding default names for environment files. & '.projectenv':'.projectenv' \\ \cline{4-4} 
	& & & 'packageenv':'packageenv' \\ \cline{4-4} 
	& & & 'personalenv':'personalenv' \\ \cline{4-4}
	& & & 'envtag':'environ' \\ \hline
	\rownumber & trace\_env & List of enviornment variables to trace & None \\ \hline
	\rownumber & logclass & If provided the string will be used for trace naming. & None \\ \hline
	\rownumber & logger & If set to True and logclass=None, use Python getChild to set trace name. & None \\ \hline
	
\end{tabularx}
\caption{Environ signature arguments}
\label{table:environ-args}
\end{table}

In addition to file name overrides, configuration dictionary can provide syntax used in these files and the environment root tag under which environment variables are defined. 

Within derivative articles environment can be updated with environment variable as follows:

\subsection{Updating environment variables} \index{Update} \index{EnvVar}
\begin{lstlisting}
env.update_env([
  EnvVar(name='REJ_ALLOWED',cast='integer',value=0,input=True),
  EnvVar(name='OUT_FILE',value='${VAR_LOC}/summary.csv',cast='path', input=True),
  EnvVar(name='RATE',override='True',cast='integer',value=5,input=True)])
\end{lstlisting}

\paragraph {if input is set to True} the variable update will be ignored if the variable is defined in parent environment. If variable is not defined in parent environment, it will be defined and set to value from the command. 
\paragraph {if input is set to False} update will overwrite variable value if variable exists, if variable is not defined it will define it.
\paragraph{override} flags environment variable as changeable by derivative program articles.

\subsection{Accessing environment variables} \index{Environment!Using}\index{get}
\begin{lstlisting}
import projenv 
env=projenv.Environ()
env.update_env([
EnvVar(name='REJ_ALLOWED',cast='integer',value=0,input=True),
EnvVar(name='OUT_FILE',value='${VAR_LOC}/summary.csv',cast='path', input=True),
EnvVar(name='RATE',override='True',cast='integer',value=5,input=True)])

ofile=env['OUT_FILE']
rate=env.get('RATE')
\end{lstlisting}

In the first case(ofile variable), direct access, KeyError exception may be sent if variable name does not exist.

In the second example(rate variable), None value will be returned if not found. 

\chapter{Environment Tree} \index{Tree}
Environment files are evaluated in hierarchical way.  The project tree and its packages are treated as nodes in a tree.  Each node can be evaluated and have its own representation of the environment.
 
\section{Single Project Environment Tree}\index{Tree!Single Project}
At each node, environment is evaluated in the following sequence:
\begin{enumerate}
	\item First .projectenv, if available, is read and set.
	\item Next packageenv, if available, is read and set.
	\item Finally, personalenv overrides, if available, is read and set.
\end{enumerate} 

\begin{figure}[h]
	\centering
	\framebox[\textwidth]{
		\begin{minipage}{0.9\textwidth}
			\dirtree{%
				.1 Project.
				.2 .projectenv.
				.2 packageenv.
				.2 personalenv.
				.2 Program A.
				.2 Package A.
				.3 packageenv.
				.3 personalenv.
				.4 Package AB.
				.5 packageenv.
				.5 personalenv.
				.5 Program AB.
			}
		\end{minipage}
	}
	\caption{Environment tree example}
	\label{fig:environment_tree}
\end{figure}

Figure~\ref{fig:environment_tree} (page~\pageref{fig:environment_tree}) shows example environment tree in a project.

When the above command is engaged in Program A, it would include environment setting of Project and Package A locations.  Program AB will include Program A, Package A and Package AB accordingly.

\lstset{
	basicstyle=\ttfamily\footnotesize,
	columns=fullflexible,
	showstringspaces=false,
	numbers=left,                   % where to put the line-numbers
	numberstyle=\tiny\color{gray},  % the style that is used for the line-numbers
	stepnumber=1,
	numbersep=5pt,                  % how far the line-numbers are from the code
	backgroundcolor=\color{white},      % choose the background color. You must add \usepackage{color}
	showspaces=false,               % show spaces adding particular underscores
	showstringspaces=false,         % underline spaces within strings
	showtabs=false,                 % show tabs within strings adding particular underscores
	frame=none,                   % adds a frame around the code
	rulecolor=\color{black},        % if not set, the frame-color may be changed on line-breaks within not-black text (e.g. commens (green here))
	tabsize=2,                      % sets default tabsize to 2 spaces
	captionpos=b,                   % sets the caption-position to bottom
	breaklines=true,                % sets automatic line breaking
	breakatwhitespace=false,        % sets if automatic breaks should only happen at whitespace
	title=\lstname,                   % show the filename of files included with \lstinputlisting;
	% also try caption instead of title  
	commentstyle=\color{gray}\upshape
}


\lstdefinelanguage{XML}
{
	morestring=[s][\color{mauve}]{"}{"},
	morestring=[s][\color{black}]{>}{<},
	morecomment=[s]{<?}{?>},
	morecomment=[s][\color{dkgreen}]{<!--}{-->},
	stringstyle=\color{black},
	identifierstyle=\color{DarkBlue},
	keywordstyle=\color{red},
	morekeywords={xmlns,xsi,noNamespaceSchemaLocation,type,id,x,y,source,target,version,tool,transRef,roleRef,objective,eventually,name,value,cast,export,input, environ}% list your attributes here
}
%\lstinputlisting[language=XML]{'sources/environ_example_1.xml'}

\begin{lstlisting}[language=XML, label=lst:project_env, caption=Example for project environment file]
<environment>
  <environ>
    <var name='AC_WS_LOC' value='${HOME}/sand/myproject' export='True'/>
    <var name='AC_ENV_NAME' value='test' export='True'/>
    <var name='AC_VAR_BASE' value='${HOME}/var/data/' export='True'/>
    <var name='AC_LOG_LEVEL' value='DEBUG' export='True'/> 
    <var name='AC_LOG_STDOUT' value='True' override='True' export='True' cast='boolean'/>
    <var name='AC_LOG_STDOUT_LEVEL' value='INFO' override='True' export='True'/>
    <var name='AC_LOG_STDERR' value='True' override='True' export='True' cast='boolean'/>
    <var name='AC_LOG_STDERR_LEVEL' value='CRITICAL' override='True' export='True'/>
  </environ>
</environment>
\end{lstlisting}

Listing \ref{lst:project_env} shows example of an environment file.  Core environment is tagged under \emph{\textless~environ\textgreater}.  Environ mechanism would look for this tag.  Once found, it would evaluate its content as environment directive.

\paragraph{Note:} \emph{\textless~environment\textgreater} tag is to provide enclosure to environ.  Environ mechanism is not depending on its existent per se.  However, some kind on enclosure is required;  \emph{\textless~environ\textgreater} can not be in top level of the XML.

\section{Multiple Project Environment Tree}\index{Tree!Multiple Projects}
At each import, environment is evaluated in the following sequence:
\begin{enumerate}
	\item First get the node representation of imported path.
	\item Evaluate it recursively (loading imports).
	\item Finally, insert the resulted imported map instead of the import directive (flat).
\end{enumerate} 

%\lstinputlisting[language=XML]{'sources/environ_example_1.xml'}

\begin{lstlisting}[language=XML, label=lst:env_proja, caption='Project A: /Users/me/projs/proja/.projectenv.xml]
<environment>
  <environ>
    <var name='FILE_LOC' value='/Users/me/tmp/' export='True'/>
    <var name='FILE_NAME' value='aname' export='True'/>
    <var name='FILE_PATH' value='${FILE_LOC}${FILE_NAME}' export='True'/>
  </environ>
</environment>
\end{lstlisting}

\begin{lstlisting}[language=XML, label=lst:env_projb, caption='Project B: /Users/me/projs/projb/.projectenv.xml']
<environment>
  <environ>
    <import name='proja' path='/Users/me/projs/proja/.projectenv.xml'/>
    <var name='FILE_NAME' value='bname' export='True'/>
  </environ>
</environment>
\end{lstlisting}

Listings \ref{lst:env_projb} shows import project directive within project B's environment.  In project B's context, FILE\_PATH will result with the value \emph{/Users/me/tmp/bname}.

\paragraph{Recursive} inclusion of environments (recursive import statement) would cause evaluation of environment variables to be loaded recursively.  Consideration is given to overrides in post import environments.

\paragraph{Note:}import path can only include environment variables that are in the OS level pre-eveluation.

\chapter{Best Practices}\index{Best Practices}
So many options, so what should one do?

\section{Naming Parameters}\index{Naming convention}
\paragraph{Project Prefix}\index{Naming convention!Prefix}
Prefix your parameters with an identifier.  Specifically if your projects would need to cooperate (import their environment).  In Listings~\ref{lst:dot_projectenv_bp} (page~\pageref{lst:dot_projectenv_bp}), we have all parameters us 'AC\_' as prefix.  We also define 'AC\_PROJ\_PREFIX' that can be used in program to construct parameter name.

\paragraph{Style}\index{Naming convention!Style} We recommend following UNIX convention for environment variables.  Use uppercase letters separated with underscore.  We use this style in all of this document listings.

\paragraph{Drivers and Derivatives}\index{Parameter!drivers}\index{Parameter!derivatives}
For the sake of this discussion we define three types of parameters:
\begin{enumerate}
	\item \em{standalone} is a parameter that is not dependent on another and is not used by another parameter.
	\item \em{driver} is a parameter that other parameters defined by it.
	\item \em{derivative} is a parameter that includes a driver in its definitions.
\end{enumerate}

A parameter can be both a driver and derivative.

Use drivers and derivative parameter definition in such a way that users may personalize the behavior of the system.  For example, developers may want to change their own directory structure to fit their own tools.
 

\section{.projectenv}\index{.projectenv}
Dot (.) projectenv usually contains parameters that are good for the all projects.  You can look at is as your standard parameters to all projects that you produce. In listings ~\ref{lst:dot_projectenv_bp}, locations are defined as derivatives of AC\_VAR\_BASE. This is useful since users of this project can override that parameter to change to their own structure.

\begin{lstlisting}[language=XML, label=lst:dot_projectenv_bp, caption='.projectenv.xml example']
<environment>
<environ>
<var name='AC_PROJ_PREFIX' value='AC_' export='True' override='True'/>
<var name='AC_VAR_BASE' value='/var/accord/data/' override='True' export='True'/>
<var name='AC_ENV_NAME' value='.' override='True' export='True'/>
<var name='AC_VAR_LOC' value='${AC_VAR_BASE}${AC_ENV_NAME}/' override='True' export='True'/>
<var name='AC_LOG_LOC' value='${AC_VAR_LOC}/log/' override='True' export='True'/>
<var name='AC_REJ_LOC' value='${AC_VAR_LOC}/rej/' override='True' export='True'/>
<var name='AC_RUN_LOC' value='${AC_VAR_LOC}/run/' override='True' export='True'/>
<var name='AC_IN_LOC' value='${AC_VAR_LOC}/in/' override='True' export='True'/>
<var name='AC_OUT_LOC' value='${AC_VAR_LOC}/out/' override='True' export='True'/>
</environ>
</environment>
\end{lstlisting}

\section{packageenv}\index{packageenv}
Packageenv includes definitions for that are specific to the project or the package.  Usually this is kept for things like RPC\_PORT or maybe MAIL\_SEND\_SMTP. 

\section{personalenv}\index{personaleenv}
Personalenv provides means to personalize an environment.  Users can override packageenv or .projectenv parameters.
you may want to exclude \em{personalenv} from your code repository (e.g., add personalenv.xml to .gitignore).  Otherwise, users wmay override each other personalizations. 

\chapter{Installation, validation and example program}\index{Installation and validation}
How to install, validate installation and use the package?

\section{Installation}\index{Installation}

To install run following command:
pip install projenv

\section{Validation}\index{Installation and validation!Validation} 
Add instruction to run test.py and check unit test cases

\section{Example}\index{Installation and validation!Example} 
See example of the program using projenv on Github
https://github.com/Acrisel/projenv/blob/master/environ/example/example.py

\end{document}